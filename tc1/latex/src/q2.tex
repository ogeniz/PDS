\section*{2.a Questão}
\label{sec:q2}

Um sistema linear, causal e invariante no tempo é descrito pela seguinte equação 
diferença:
$$
y[n] - \dfrac{1}{2}y[n - 1] + \dfrac{1}{4}y[n - 2] = x[n] + 2x[n - 1] + x[n - 3]
$$

\begin{enumerate}
  \item Desenvolva um algoritmo recursivo para calcular a resposta ao impulso do
    sistema, para $0 \leq n \leq 100$, e plote-a utilizando a função do Matlab
    \emph{stem}.
  \item Faça o mesmo que o ítem anterior, porém agora utilizando a função
    \emph{filter} do Matlab.
  \item Determine se o sistema é estável ou não analisando a resposta ao impulso.
    Explique o porquê.
  \item Se a entrada do sistema for a seqüência
    $x[n]$ $=$ $\{5 + 3\cos(0.2\pi n) + 4\sin(0.6\pi n)\} u[n]$, determine a saída $y[n]$
    para $0 \leq n \leq 200$ utilizando sua solução recursiva e compare-a com o
    resultado obtido utilizando a função filter.
\end{enumerate}

%%% Local Variables:
%%% mode: latex
%%% TeX-master: "../main"
%%% End: